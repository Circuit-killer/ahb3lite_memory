\includegraphics[width=4.96389in,height=2.32778in]{media/image1.png}

\section{Introduction}\label{introduction}

The Roa Logic AHB-Lite Memory IP is a fully parameterized soft IP
implementing on-chip memory for access by an AHB-Lite based Master. All
signals defined in the \emph{AMBA 3 AHB-Lite v1.0} specifications are
fully supported.

The IP supports a single AHB-Lite based host connection and enables
address \& data widths, memory depth \& target technology to be
specified via parameters. An option to register the memory output is
also provided.

\protect\hypertarget{_Toc346441234}{}{\protect\hypertarget{_Toc347141354}{}{}}\includegraphics[width=3.18056in,height=2.00000in]{media/image2.emf}\\[2\baselineskip]Figure
: AHB-Lite Memory System

\subsection{Features}\label{features}

\begin{itemize}
\item
  Full support for AMBA 3 AHB-Lite protocol
\item
  Fully parameterized
\item
  User-defined address and byte-aligned data widths supported
\item
  Configurable memory depth, limited only by target technology
  capability
\item
  Technology-specific memory cells instantiated automatically
\item
  Combinatorial or registered data output
\end{itemize}

Table of Contents

Introduction 2

Features 2

1 Getting Started 4

1.1 Deliverables 4

1.2 Running the testbench 5

1.2.1 Self-checking testbench 5

1.2.2 Makefile setup 5

1.2.3 Makefile backup 5

1.2.4 No Makefile 5

2 Specifications 6

2.1 Functional Description 6

3 Configurations 7

3.1 Introduction 7

3.1 Core Parameters 7

3.1.1 MEM\_DEPTH 7

3.1.2 HADDR\_SIZE 7

3.1.3 HDATA\_SIZE 7

3.1.4 TECHNOLOGY 8

3.1.5 REGISTERED\_OUTPUT 8

4 Interfaces 9

4.1 AHB-Lite Interface 9

4.1.1 HRESETn 9

4.1.2 HCLK 9

4.1.3 HSEL 9

4.1.4 HTRANS 10

4.1.5 HADDR 10

4.1.6 HWDATA 10

4.1.7 HRDATA 10

4.1.8 HWRITE 10

4.1.9 HSIZE 10

4.1.10 HBURST 11

4.1.11 HPROT 11

4.1.12 HREADYOUT 11

4.1.13 HREADY 11

4.1.14 HRESP 11

5 Resources 12

6 Revision History 13

\section{Getting Started}\label{getting-started}

\subsection{Deliverables}\label{deliverables}

\includegraphics[width=1.60000in,height=3.82222in]{media/image3.emf}All
IP is delivered as a zipped tarball, which can be unzipped with all
common compression tools (like unzip, winrar, tar, \ldots{}).

The tarball contains a directory structure as outlined below.

The \emph{doc} directory contains relevant documents like user guides,
application notes, and datasheets.

The \emph{rtl} directory contains the actual IP design files. Depending
on the license agreement the AHB-Lite Memory is delivered as either
encrypted Verilog-HDL or as plain SystemVerilog source files. Encrypted
files have the extension ``.enc.sv'', plain source files have the
extension ``.sv''. The files are encryption according to the IEEE-P1735
encryption standard. Encryption keys for Mentor Graphics (Modelsim,
Questasim, Precision), Synplicity (Synplify, Synplify-Pro), and Aldec
(Active-HDL, Riviera-Pro) are provided. As such there should be no issue
targeting any existing FPGA technology.

If any other synthesis or analysis tool is used then a plain source RTL
delivery may be needed. A separate license agreement and NDA is required
for such a delivery.

The \emph{bench} directory contains the (encrypted) source files for the
testbench.

The \emph{sim} directory contains the files/structure to run the
simulations. Section 1.2 `Running the testbench' provides for
instructions on how to use the makefile.

\subsection{Running the testbench}\label{running-the-testbench}

The AHB-Lite Memory IP comes with a dedicated testbench that tests all
features of the design and finally runs a full random test. The
testbench is started from a Makefile that is provided with the IP.

The Makefile is located in the
\textless{}\emph{install\_dir}\textgreater{}/sim/rtlsim/run directory.
The Makefile supports most commonly used simulators; Modelsim/Questasim,
Cadence ncsim, Aldec Riviera, and Synopsys VCS.

To start the simulation, enter the
\textless{}\emph{install\_dir}\textgreater{}/sim/rtlsim/run directory
and type: \textbf{make \textless{}\emph{simulator}\textgreater{}}. Where
simulator is any of: msim (for modelsim/questasim), ncsim (for Cadence
ncsim), riviera (for Aldec Riviera-Pro), or vcs (for Synopsys VCS). For
example type \textbf{make msim} to start the testbench in
Modelsim/Questasim.

\subsubsection{Self-checking testbench}\label{self-checking-testbench}

The testbenches is a self-checking testbench intended to be executed
from the command line. There is no need for a GUI or a waveform viewer.
Once the testbench completes it displays a summary and closes the
simulator.

\subsubsection{Makefile setup}\label{makefile-setup}

The simulator is executed in its associated directory. Inside this
directory is another Makefile that contains simulator specific commands
to start and execute the simulation. The
\textless{}\emph{install\_dir}\textgreater{}/sim/rtlsim/run/Makefile
enters the correct directory and calls the simulator specific Makefile.

For example modelsim is executed in the
\textless{}\emph{install\_dir}\textgreater{}/sim/rtlsim/run/msim
directory. Typing \textbf{make msim} loads the main Makefile, which then
enters the msim sub-directory and calls its Makefile. This Makefile
contains commands to compile the RTL and testbench sources with
Modelsim, start the Modelsim simulator, and run the simulation.

\subsubsection{Makefile backup}\label{makefile-backup}

The \textless{}\emph{install\_dir}\textgreater{}/sim/rtlsim/bin
directory contains backups of the original Makefiles. It may be
desirable to modify or extend the Makefiles or to completely clean the
run directory. Use the backups to restore the original setup.

\subsubsection{No Makefile}\label{no-makefile}

For users unfamiliar with Makefiles or those on systems that do not
natively support make (e.g. Windows) a run.do file is provided that can
be used with Modelsim/Questasim and Riviera-Pro.

\section{Specifications}\label{specifications}

\subsection{Functional Description}\label{functional-description}

\protect\hypertarget{_Toc326677726}{}{\protect\hypertarget{_Ref285296883}{}{\protect\hypertarget{_Ref285296868}{}{\protect\hypertarget{_Ref285296852}{}{\protect\hypertarget{_Ref285296837}{}{\protect\hypertarget{_Ref285296804}{}{}}}}}}\includegraphics[width=2.85556in,height=3.25833in]{media/image4.emf}The
AHB-Lite Memory IP is a flexible, fully configurable, IP that enables
designers to attach internal device memory to AHB-Lite based host. The
width and depth of the memory, together with an optional registered
output stage, are specified via parameters.

The IP is designed to easily support a wide range of target
technologies, automatically implementing technology-specific memory
cells according to the chosen target. A generic behavioural
implementation is also supported.

\subsection{AHB-Lite Bus Locking
Support}\label{ahb-lite-bus-locking-support}

The \emph{AMBA 3 AHB-Lite v1.0} protocol supports bus locking. Typically
a locked transfer is used to ensure that a slave does not perform other
operations between the read and write phases of a transaction. Given the
AHB-Lite Memory IP performs no such operations, bus locking is not
supported and does not provide the HMASTLOCK input associated with this
capability

\section{Configurations}\label{configurations}

\subsection{Introduction}\label{introduction-1}

\protect\hypertarget{_Toc346441248}{}{}The size and implementation style
of the memory is defined via HDL parameters. These are specified in the
following section.

\subsection{Core Parameters}\label{core-parameters}

\begin{longtable}[]{@{}llll@{}}
\toprule
Parameter & Type & Default & Description\tabularnewline
\midrule
\endhead
MEM\_DEPTH & Integer & 256 & Memory Depth (Words)\tabularnewline
HADDR\_SIZE & Integer & 32 & Address Bus Size (Bits)\tabularnewline
HDATA\_SIZE & Integer & 32 & Data Bus Size (Bits)\tabularnewline
TECHNOLOGY & String & GENERIC & Implementation Technology\tabularnewline
REGISTERED\_OUTPUT & String & NO & Is output registered?\tabularnewline
\bottomrule
\end{longtable}

\protect\hypertarget{_Toc326677729}{}{}Table 3‑1: AHB-Lite Memory
Parameters

\subsubsection{MEM\_DEPTH}\label{mem_depth}

MEM\_DEPTH defines the depth of the memory -- i.e. number of HDATA\_SIZE
words to be stored. The maximum depth supported is dependent upon the
target technology chosen.

\subsubsection{HADDR\_SIZE}\label{haddr_size}

The HADDR\_SIZE parameter specifies the address bus size to connect to
the AHB-Lite based host. The maximum size supported is 32 bits.

\subsubsection{HDATA\_SIZE}\label{hdata_size}

The HDATA\_SIZE parameter specifies the data bus size to connect to the
AHB-Lite based host. The maximum size supported is 32 bits.

\protect\hypertarget{_Toc346451087}{}{\protect\hypertarget{_Toc347141370}{}{\protect\hypertarget{_Ref347483875}{}{}}}

\subsubsection{TECHNOLOGY}\label{technology}

The TECHNOLOGY parameter defines the target silicon technology and may
be one of the following values:

\begin{longtable}[]{@{}ll@{}}
\toprule
Parameter Value & Description\tabularnewline
\midrule
\endhead
GENERIC & Behavioural Implementation\tabularnewline
N3X & eASIC Nextreme-3 Structured ASIC\tabularnewline
N3XS & eASIC Nextreme-3S Structured ASIC\tabularnewline
\bottomrule
\end{longtable}

Table 3‑2: Supported Technology Targets

Details of the implementations corresponding to these parameter values
can be found in Section 6, Technology Support

\subsubsection{REGISTERED\_OUTPUT}\label{registered_output}

The REGISTERED\_OUTPUT parameter defines if the output of the memory is
registered on assertion of the HREADY signal. It is specified as `YES'
or `NO' (default).

\section{Interfaces}\label{interfaces}

\subsection{AHB-Lite Interface}\label{ahb-lite-interface}

The AHB-Lite interface is a regular AHB-Lite slave port. All signals are
supported. See the \emph{AMBA 3 AHB-Lite Specification} for a complete
description of the signals.

\begin{longtable}[]{@{}llll@{}}
\toprule
Port & Size & Direction & Description\tabularnewline
\midrule
\endhead
HRESETn & 1 & Input & Asynchronous active low reset\tabularnewline
HCLK & 1 & Input & Clock Input\tabularnewline
HSEL & 1 & Input & Bus Select\tabularnewline
HTRANS & 2 & Input & Transfer Type\tabularnewline
HADDR & HADDR\_SIZE & Input & Address Bus\tabularnewline
HWDATA & HDATA\_SIZE & Input & Write Data Bus\tabularnewline
HRDATA & HDATA\_SIZE & Output & Read Data Bus\tabularnewline
HWRITE & 1 & Input & Write Select\tabularnewline
HSIZE & 3 & Input & Transfer Size\tabularnewline
HBURST & 3 & Input & Transfer Burst Size\tabularnewline
HPROT & 4 & Input & Transfer Protection Level\tabularnewline
HREADYOUT & 1 & Output & Transfer Ready Output\tabularnewline
HREADY & 1 & Input & Transfer Ready Input\tabularnewline
HRESP & 1 & Input & Transfer Response\tabularnewline
\bottomrule
\end{longtable}

\protect\hypertarget{_Toc326677769}{}{}Table 4‑1: AHB-Lite Interface
Ports

\subsubsection{HRESETn}\label{hresetn}

When the active low asynchronous HRESETn input is asserted (`0'), the
interface is put into its initial reset state.

\subsubsection{HCLK}\label{hclk}

HCLK is the interface system clock. All internal logic for the AMB3-Lite
interface operates at the rising edge of this system clock and AHB bus
timings are related to the rising edge of HCLK.

\subsubsection{HSEL}\label{hsel}

The AHB-Lite interface only responds to other signals on its bus when
HSEL is asserted (`1'). When HSEL is negated (`0') the interface
considers the bus IDLE and negates HREADYOUT (`0').

\protect\hypertarget{_Toc346441257}{}{\protect\hypertarget{_Toc346209448}{}{}}

\subsubsection{HTRANS}\label{htrans}

HTRANS indicates the type of the current transfer.

\begin{longtable}[]{@{}lll@{}}
\toprule
HTRANS & Type & Description\tabularnewline
\midrule
\endhead
00 & IDLE & No transfer required\tabularnewline
01 & BUSY & Connected master is not ready to accept data, but intents to
continue the current burst.\tabularnewline
10 & NONSEQ & First transfer of a burst or a single
transfer\tabularnewline
11 & SEQ & Remaining transfers of a burst\tabularnewline
\bottomrule
\end{longtable}

Table 4‑2: AHB-Lite Transfer Type (HTRANS)

\subsubsection{HADDR}\label{haddr}

HADDR is the address bus. Its size is determined by the HADDR\_SIZE
parameter and is driven to the connected peripheral.

\subsubsection{HWDATA}\label{hwdata}

HWDATA is the write data bus. Its size is determined by the HDATA\_SIZE
parameter and is driven to the connected peripheral.

\subsubsection{HRDATA}\label{hrdata}

HRDATA is the read data bus. Its size is determined by HDATA\_SIZE
parameter and is sourced by the APB4 peripheral.

\subsubsection{HWRITE}\label{hwrite}

HWRITE is the read/write signal. HWRITE asserted (`1') indicates a write
transfer.

\subsubsection{HSIZE}\label{hsize}

HSIZE indicates the size of the current transfer.

\begin{longtable}[]{@{}lll@{}}
\toprule
HSIZE & Size & Description\tabularnewline
\midrule
\endhead
000 & 8bit & Byte\tabularnewline
001 & 16bit & Half Word\tabularnewline
010 & 32bit & Word\tabularnewline
011 & 64bits & Double Word\tabularnewline
100 & 128bit &\tabularnewline
101 & 256bit &\tabularnewline
110 & 512bit &\tabularnewline
111 & 1024bit &\tabularnewline
\bottomrule
\end{longtable}

\protect\hypertarget{_Toc326677773}{}{}Table 4‑3: Transfer Size Values
(HSIZE)

\protect\hypertarget{_Toc346441263}{}{\protect\hypertarget{_Toc346209454}{}{}}

\subsubsection{HBURST}\label{hburst}

HBURST indicates the transaction burst type -- a single transfer or part
of a burst.

\begin{longtable}[]{@{}lll@{}}
\toprule
HBURST & Type & Description\tabularnewline
\midrule
\endhead
000 & SINGLE & Single access\tabularnewline
001 & INCR & Continuous incremental burst\tabularnewline
010 & WRAP4 & 4-beat wrapping burst\tabularnewline
011 & INCR4 & 4-beat incrementing burst\tabularnewline
100 & WRAP8 & 8-beat wrapping burst\tabularnewline
101 & INCR8 & 8-beat incrementing burst\tabularnewline
110 & WRAP16 & 16-beat wrapping burst\tabularnewline
111 & INCR16 & 16-beat incrementing burst\tabularnewline
\bottomrule
\end{longtable}

\protect\hypertarget{_Toc326677774}{}{}Table 4‑4: AHB-Lite Burst Types
(HBURST)

\subsubsection{HPROT}\label{hprot}

The HPROT signals provide additional information about the bus transfer
and are intended to implement a level of protection.

\begin{longtable}[]{@{}lll@{}}
\toprule
Bit\# & Value & Description\tabularnewline
\midrule
\endhead
3 & 1 & Cacheable region addressed\tabularnewline
& 0 & Non-cacheable region addressed\tabularnewline
2 & 1 & Bufferable\tabularnewline
& 0 & Non-bufferable\tabularnewline
1 & 1 & Privileged Access\tabularnewline
& 0 & User Access\tabularnewline
0 & 1 & Data Access\tabularnewline
& 0 & Opcode fetch\tabularnewline
\bottomrule
\end{longtable}

\protect\hypertarget{_Toc326677775}{}{}Table 4‑5: AHB-Lite Transaction
Protection Signals (HPROT)

\subsubsection{HREADYOUT}\label{hreadyout}

HREADYOUT indicates that the current transfer has finished.

\subsubsection{HREADY}\label{hready}

HREADY indicates whether or not the addressed peripheral is ready to
transfer data. When HREADY is negated (`0') the peripheral is not ready,
forcing wait states. When HREADY is asserted (`1') the peripheral is
ready and the transfer completed.

\subsubsection{HRESP}\label{hresp}

HRESP is the instruction transfer response and indicates OKAY (`0') or
ERROR (`1'). An error response causes an Instruction Bus Error
Interrupt.

\section{Resources}\label{resources}

Below are some example implementations for various platforms.

All implementations are push button, no effort has been undertaken to
reduce area or improve performance.

\begin{longtable}[]{@{}lllll@{}}
\toprule
Platform & DFF & Logic Cells & Memory & Performance (MHz)\tabularnewline
\midrule
\endhead
& & & &\tabularnewline
& & & &\tabularnewline
& & & &\tabularnewline
\bottomrule
\end{longtable}

Table 5‑1: Resource Utilization Examples

\section{Technology Support}\label{technology-support}

Physical memory implementation in silicon depends on the target
technology chosen. The AHB-Lite Memory IP allows a designer to specify
either a generic (i.e. behavioural) implementation or one of multiple
technology-specific implementations via the TECHNOLOGY parameter (see
section 0).

This section provides details of these implementations

\subsection{GENERIC Implementation}\label{generic-implementation}

The GENERIC option is used to implement regular behavioural HDL allowing
both the physical implementation to be controlled during hardware
synthesis and full behavioural simulation to be performed.

\subsection{eASIC Structured ASIC
Support}\label{easic-structured-asic-support}

The IP supports the Nextreme-3 and Nextreme-3S families as described
below. Please refer to the relevant technology datasheets for complete
details of the memory structures referenced.

\subsubsection{Nextreme-3 Implementation
(N3X)}\label{nextreme-3-implementation-n3x}

The Nextreme-3 family of devices features 9Kbit memory blocks, referred
to as `bRAM'. When the TECHNOLOGY parameter is defined as `N3X', all
memory will be implemented using 9Kbit bRAM cells.

\subsubsection{Nextreme-3S Implementation
(N3XS)}\label{nextreme-3s-implementation-n3xs}

The Nextreme-3S series of devices features separate 2Kbit and 18Kbit
memory blocks, referred to as `bRAM2K' and `bRAM18K' respectively. The
choice of which of these blocks are implemented when the TECHNOLOGY
parameter is defined as `N3XS' is as follows:

\begin{longtable}[]{@{}lll@{}}
\toprule
TECHNOLOGY & Memory Size & Implementation\tabularnewline
\midrule
\endhead
N3XS & ≤4096 bits & bRAM2K blocks only\tabularnewline
N3XS & \textgreater{}4096 bits & bRAM18K blocks only\tabularnewline
\bottomrule
\end{longtable}

Table ‑: Nextreme-3S Memory Implementation Styles

\section{Revision History}\label{revision-history}

\begin{longtable}[]{@{}lll@{}}
\toprule
Date & Rev. & Comments\tabularnewline
\midrule
\endhead
01-Feb-2017 & 1.0 &\tabularnewline
& &\tabularnewline
& &\tabularnewline
& &\tabularnewline
\bottomrule
\end{longtable}

Table 7‑1: Revision History
