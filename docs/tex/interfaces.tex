\chapter{Interfaces}\label{interfaces}

\section{AHB-Lite Interface}\label{ahb-lite-interface}

The AHB-Lite interface is a regular AHB-Lite slave port. All signals are
supported. See the \emph{AMBA 3 AHB-Lite Specification} for a complete
description of the signals.

\begin{longtable}[]{@{}lcll@{}}
	\toprule
	Port      & Size        & Direction & Description\tabularnewline
	\midrule
	\endhead
		\texttt{HRESETn}   & 1                    & Input  & Asynchronous active low reset\tabularnewline
		\texttt{HCLK}      & 1                    & Input  & Clock Input\tabularnewline
		\texttt{HSEL}      & 1                    & Input  & Bus Select\tabularnewline
		\texttt{HTRANS}    & 2                    & Input  & Transfer Type\tabularnewline
		\texttt{HADDR}     & \texttt{HADDR\_SIZE} & Input  & Address Bus\tabularnewline
		\texttt{HWDATA}    & \texttt{HDATA\_SIZE} & Input  & Write Data Bus\tabularnewline
		\texttt{HRDATA}    & \texttt{HDATA\_SIZE} & Output & Read Data Bus\tabularnewline
		\texttt{HWRITE}    & 1                    & Input  & Write Select\tabularnewline
		\texttt{HSIZE}     & 3                    & Input  & Transfer Size\tabularnewline
		\texttt{HBURST}    & 3                    & Input  & Transfer Burst Size\tabularnewline
		\texttt{HPROT}     & 4                    & Input  & Transfer Protection Level\tabularnewline
		\texttt{HREADYOUT} & 1                    & Output & Transfer Ready Output\tabularnewline
		\texttt{HREADY}    & 1                    & Input  & Transfer Ready Input\tabularnewline
		\texttt{HRESP}     & 1                    & Input  & Transfer Response\tabularnewline
	\bottomrule
	\caption{AHB-Lite Interface	Ports}
\end{longtable}

\subsection{HRESETn}\label{hresetn}

When the active low asynchronous \texttt{HRESETn} input is asserted (`0'), the
interface is put into its initial reset state.

\subsection{HCLK}\label{hclk}

\texttt{HCLK} is the interface system clock. All internal logic for the AMB3-Lite
interface operates at the rising edge of this system clock and AHB bus
timings are related to the rising edge of \texttt{HCLK}.

\subsection{HSEL}\label{hsel}

The AHB-Lite interface only responds to other signals on its bus when
\texttt{HSEL} is asserted (`1'). When HSEL is negated (`0') the interface
considers the bus IDLE and negates \texttt{HREADYOUT} (`0').

\pagebreak

\subsection{HTRANS}\label{htrans}

HTRANS indicates the type of the current transfer.

\begin{longtable}[]{@{}llp{11cm}@{}}
	\toprule
		HTRANS & Type   & Description\tabularnewline
	\midrule
	\endhead
		00     & IDLE   & No transfer required\tabularnewline
		01     & BUSY   & Connected master is not ready to accept data, but intents to continue the current burst.\tabularnewline
		10     & NONSEQ & First transfer of a burst or a single transfer\tabularnewline
		11     & SEQ    & Remaining transfers of a burst\tabularnewline
	\bottomrule
	\caption{AHB-Lite Transfer Type (HTRANS)}
\end{longtable}

\subsection{HADDR}\label{haddr}

\texttt{HADDR} is the address bus. Its size is determined by the \texttt{HADDR\_SIZE}
parameter and is driven to the connected peripheral.

\subsection{HWDATA}\label{hwdata}

\texttt{HWDATA} is the write data bus. Its size is determined by the \texttt{HDATA\_SIZE}
parameter and is driven to the connected peripheral.

\subsection{HRDATA}\label{hrdata}

\texttt{HRDATA} is the read data bus. Its size is determined by \texttt{HDATA\_SIZE}
parameter and is sourced by the APB4 peripheral.

\subsection{HWRITE}\label{hwrite}

\texttt{HWRITE} is the read/write signal. \texttt{HWRITE} asserted (`1') indicates a write
transfer.

\subsection{HSIZE}\label{hsize}

\texttt{HSIZE} indicates the size of the current transfer.

\begin{longtable}[]{@{}cll@{}}
	\toprule
	HSIZE & Size & Description\tabularnewline
	\midrule
	\endhead
	\texttt{000} & 8bit & Byte\tabularnewline
	\texttt{001} & 16bit & Half Word\tabularnewline
	\texttt{010} & 32bit & Word\tabularnewline
	\texttt{011} & 64bits & Double Word\tabularnewline
	\texttt{100} & 128bit &\tabularnewline
	\texttt{101} & 256bit &\tabularnewline
	\texttt{110} & 512bit &\tabularnewline
	\texttt{111} & 1024bit &\tabularnewline
	\bottomrule
	\caption{HSIZE Values}
\end{longtable}

\subsection{HBURST}\label{hburst}

\texttt{HBURST} indicates the transaction burst type -- a single transfer or part
of a burst.

\begin{longtable}[]{@{}cll@{}}
	\toprule
		HBURST & Type & Description\tabularnewline
	\midrule
	\endhead
		\texttt{000} & SINGLE & Single access\tabularnewline
		\texttt{001} & INCR   & Continuous incremental burst\tabularnewline
		\texttt{010} & WRAP4  & 4-beat wrapping burst\tabularnewline
		\texttt{011} & INCR4  & 4-beat incrementing burst\tabularnewline
		\texttt{100} & WRAP8  & 8-beat wrapping burst\tabularnewline
		\texttt{101} & INCR8  & 8-beat incrementing burst\tabularnewline
		\texttt{110} & WRAP16 & 16-beat wrapping burst\tabularnewline
		\texttt{111} & INCR16 & 16-beat incrementing burst\tabularnewline
	\bottomrule
	\caption{AHB-Lite Burst Types (HBURST)}
\end{longtable}

\subsection{HPROT}\label{hprot}

The \texttt{HPROT} signals provide additional information about the bus transfer
and are intended to implement a level of protection.

\begin{longtable}[]{@{}ccl@{}}
	\toprule
		Bit\# & Value & Description\tabularnewline
	\midrule
	\endhead
		3 & 1 & Cacheable region addressed\tabularnewline
		  & 0 & Non-cacheable region addressed\tabularnewline
		2 & 1 & Bufferable\tabularnewline
		  & 0 & Non-bufferable\tabularnewline
		1 & 1 & Privileged Access\tabularnewline
		  & 0 & User Access\tabularnewline
		0 & 1 & Data Access\tabularnewline
		  & 0 & Opcode fetch\tabularnewline
	\bottomrule
	\caption{AHB-Lite Protection Signals (HPROT)}
\end{longtable}

\subsection{HREADYOUT}\label{hreadyout}

\texttt{HREADYOUT} indicates that the current transfer has finished.

\subsection{HREADY}\label{hready}

\texttt{HREADY} indicates whether or not the addressed peripheral is ready to
transfer data. When \texttt{HREADY} is negated (`0') the peripheral is not ready,
forcing wait states. When \texttt{HREADY} is asserted (`1') the peripheral is
ready and the transfer completed.

\subsection{HRESP}\label{hresp}

\texttt{HRESP} is the instruction transfer response and indicates OKAY (`0') or
ERROR (`1'). An error response causes an Instruction Bus Error
Interrupt.